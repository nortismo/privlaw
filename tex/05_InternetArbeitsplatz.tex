\hypertarget{internet-am-arbeitsplatz-whistleblowing}{%
\section{Internet am Arbeitsplatz \&
Whistleblowing}\label{internet-am-arbeitsplatz-whistleblowing}}

\hypertarget{hauptinteresse-der-arbeitgeber}{%
\subsection{Hauptinteresse der
Arbeitgeber}\label{hauptinteresse-der-arbeitgeber}}

Mitarbeiter sind \textbf{loyal} und nutzen ihre Arbeitszeit
\textbf{überwiegend} zum Vorteil des Arbeitgebers. Mitarbeier halten
sensitive Informationen \textbf{vertraulich} und vermischen nicht
gschäftliche und private Daten.

\hypertarget{hauptinteresse-der-arbeitnehmer}{%
\subsection{Hauptinteresse der
Arbeitnehmer}\label{hauptinteresse-der-arbeitnehmer}}

\begin{itemize}
\tightlist
\item
  Zugang zu den für die Arbeit notwendigen Informationen
\item
  Nicht von der ``privaten Welt'' abgeschnitten zu sein
\item
  Respekt der Privatsphäre der Arbeitnehmers durch den Arbeitgeber
\item
  Schutz der personenbezogenen Daten durch den Arbeitgeber
\end{itemize}

Art 328B im OR Regelt das Arbeitsverhältnis und der Datenschutz.

Art. 26 ArGV 3 (Verordnung 3 zum Arbeitsgesetz):

\begin{enumerate}
\def\labelenumi{\arabic{enumi}.}
\tightlist
\item
  Überwachungs- und Kontrollsysteme, die das \textbf{Verhalten} der
  Arbeitnehmer am Arbeitsplatz überwachen sollen, \textbf{dürfen nicht
  eingesetzt werden}.\\
\item
  Sind Überwachungs- oder Kontrollsysteme aus anderen Gründen
  erforderlich, sind sie insbesondere so zu gestalten und anzuordnen,
  dass die \textbf{Gesundheit} und die \textbf{Bewegungsfreiheit} der
  Arbeitnehmer dadurch \textbf{nicht beeinträchtigt} werden.
\end{enumerate}

\hypertarget{kontrolle-der-kommunikation}{%
\subsection{Kontrolle der
Kommunikation}\label{kontrolle-der-kommunikation}}

\textbf{Permanente} \& systematische \textbf{Kontrolle} der
Kommunikation (Internet, Email, Telefon) ist \textbf{illegal}. Wenn
\textbf{vorgängig angekündigt \& zeitlich eingeschränkt}, sind
Stichproben \& individuelle Kontrollen aber \textbf{zulässig}.

\hypertarget{verpflichtung-von-whistleblowing}{%
\subsection{Verpflichtung von
Whistleblowing}\label{verpflichtung-von-whistleblowing}}

\begin{itemize}
\tightlist
\item
  Wenn ein Arbeitnehmer in seinem Verantwortungsbereich illegale
  Handlungen feststellt, so hat er seine Vorgesetzten zu informieren
  (\textbf{Treuepflichten}, Art. 321a Abs. 1 OR). Dieser Schritt lässt
  sich jedoch kaum jemandem Vorwerfen.
\item
  \textbf{Beamte (des Bundes) sind verpflichtet}, ihre Vorgesetzten über
  alle festgestellten illegale Aktivitäten zu informieren (Art. 22a
  BPG).
\item
  Wenn der WB mit seinen Feststellungen bei den Vorgesetzten keine
  Beachtung findet, so ist er zur Information der Öffentlichkeit nur
  berechtigt, wenn mehrere Voraussetzungen erfüllt sind!
  \textbf{(Proportionalität \& Kaskadenprinzip)}.
\end{itemize}

\textbf{Schutz:}\\
Zwar Schutz der Meinungsäusserungsfreiheit durch Art. 10 EMRK \& Art. 16
BV, aber in der Praxis Gefahr von strafrechtlicher Verfolgung.

\section{Internet am Arbeitsplatz}

\subsection{Hauptinteresse der Arbeitgeber}

Mitarbeiter sind \textbf{loyal} und nutzen ihre Arbeitszeit
\textbf{überwiegend} zum Vorteil des Arbeitgebers. Mitarbeiter halten
sensitive Informationen \textbf{vertraulich} und vermischen nicht
geschäftliche und private Daten.

\subsection{Hauptinteresse der Arbeitnehmer}

\begin{itemize}
\tightlist
\item Zugang zu den für die Arbeit notwendigen Informationen
\item Nicht von der ``privaten Welt'' abgeschnitten zu sein
\item Respekt der Privatsphäre der Arbeitnehmers durch den Arbeitgeber
\item Schutz der personenbezogenen Daten durch den Arbeitgeber
\end{itemize}

Art 328B im OR Regelt das Arbeitsverhältnis und der Datenschutz.

\subsection{Datenschutz \& Bewerbungsverfahren}

\textbf{Referenzauskünfte} sind nur mit \textbf{ausdrücklicher Zustimmung} des
Arbeitnehmers erlaubt.

\subsection{Überwachung des Arbeitnehmers (Art. 26 ArGV 3)}
\label{sec:IW-Überwachung}

\begin{enumerate}
	\tightlist
	\item Überwachungs- und Kontrollsysteme, die das \textbf{Verhalten} der
	Arbeitnehmer am Arbeitsplatz überwachen sollen, \textbf{dürfen nicht
	eingesetzt werden}.
	\item Sind Überwachungs- oder Kontrollsysteme aus anderen Gründen
	erforderlich, sind sie insbesondere so zu gestalten und anzuordnen,
	dass die \textbf{Gesundheit} und die \textbf{Bewegungsfreiheit} der
	Arbeitnehmer dadurch \textbf{nicht beeinträchtigt} werden.
\end{enumerate}

\subsection{Kontrolle der Kommunikation}

\textbf{Permanente} \& systematische \textbf{Kontrolle} der
Kommunikation (Internet, Email, Telefon) ist \textbf{illegal}. Wenn
\textbf{vorgängig angekündigt \& zeitlich eingeschränkt}, sind
Stichproben \& individuelle Kontrollen aber \textbf{zulässig}. Ausgenommen
davon sind strafrechtliche Untersuchungen von Behörden. Diese sind immer legal
sofern sie korrekt angeordnet wurden.

\mbox{}\\
Um allfälligen Problemen vorzubeugen ist es sinnvoll, den Arbeitnehmer einen
``Code of Conduct'' (Nutzungsreglement) unterzeichnen zu lassen.

\clearpage
\section{Whistleblowing}

\emph{Ein Whistleblower is a person who exposes any kind of information or
activity that is deemed illegal, unethical or not correct within an
organisation that is either private or public.} - Wikipedia

\mbox{}\\
Es mag illegal sein, solche Informationen aufzudecken, aber ethisch notwendig.
Das Aufdecken von solchen Informationen bedeutet grosse Risiken, sowohl für den
Whistleblower als auch für die betroffene Organisation.

\mbox{}\\
Wichtige Begriffe im Zusammenhang mit Whistleblowing sind:
\begin{description}
	\tightlist
	\item[Illegal] Verletzung einer Gesetzesbestimmung
	\item[Illegitim] Verletzung einer internen Regelung oder Direktive
	\item[Unethisch] Unmoralische \& unethische Praktiken
\end{description}

\subsection{Verpflichtung von Whistleblowing}
\label{sec:IW-Verpflichtungen}

\begin{itemize}
\tightlist
	\item Wenn ein Arbeitnehmer in seinem Verantwortungsbereich illegale
	Handlungen feststellt, so hat er seine Vorgesetzten zu informieren
	(\textbf{Treuepflichten}, Art. 321a Abs. 1 OR). Dieser Schritt lässt
	sich jedoch kaum jemandem Vorwerfen.
	\item \textbf{Beamte (des Bundes) sind verpflichtet}, ihre Vorgesetzten über
	alle festgestellten illegale Aktivitäten zu informieren (Art. 22a
	BPG).
	\item Wenn der WB mit seinen Feststellungen bei den Vorgesetzten keine
	Beachtung findet, so ist er zur Information der Öffentlichkeit nur
	berechtigt, wenn mehrere Voraussetzungen erfüllt sind!
	\textbf{(Proportionalität \& Kaskadenprinzip)}.
	\item \textbf{Corporate Governance \& Compliance/Risk Management verlangen
	ständige Sicherheitsüberprüfungen} (Art. 716a Abs. 1 OR). Eventuell sind die
	Regularien des Sarbanes-Oxley Acts (SOX) anwendbar.
	\item Es muss organisatorisch sichergestellt sein, dass der WB seine
	Anonymität behält, aber dennoch die Kommunikation mit ihm möglich ist!
\end{itemize}

\subsection{Rechtlicher Schutz des Whistleblowers}
\label{sec:IW-Schutz}
Zwar Schutz der Meinungsäusserungsfreiheit durch Art. 10 EMRK \& Art. 16
BV, aber in der Praxis Gefahr von strafrechtlicher Verfolgung. Aktuell wird
das OR überarbeitet um zusätzlichen rechtlichen Schutz für Whistleblower zu
schaffen. Die Gesetzesänderung ist aber noch nicht definitiv. Aktuell gibt
es nur geringen Schutz bei missbräuchlicher Kündigung durch eine ``Pönale'' von
Maximum 6 Monatslöhnen (Art. 336 OR).

\subsubsection{Entwurf Art. 321A OR}

Die Information der Öffentlichkeit über eine Unregelmässigkeit steht im Einklang
mit der Treuepflicht der Arbeitnehmerin oder des Arbeitnehmers, wenn:
\begin{itemize}
	\tightlist
	\item die Arbeitnehmerin oder der Arbeitnehmer ernsthafte Gründe hat, den
	gemeldeten Umstand in guten Treuen für wahr zu halten;
	\item sie oder er die Unregelmässigkeit vorgängig der zuständigen Behörde
	nach Artikel 321 oder 321 gemeldet hat; und
	\item eine der folgenden Voraussetzungen erfüllt ist:
	\begin{enumerate}
		\tightlist
		\item Die Arbeitnehmerin oder der Arbeitnehmer hat die zuständige
		Behörde ersucht, über die Behandlung der Meldung informiert zu werden
		und diese hat ihr oder ihm die geeigneten Auskünfte nicht innert
		vierzehn Tagen ab Erhalt des Ersuchens erteilt.
		\item Nach der Meldung an die Behörde wurde der Arbeitnehmerin oder dem
		Arbeitnehmer gekündigt oder sind ihr oder ihm andere Nachteile entstanden
	\end{enumerate}
\end{itemize}

\section{Grundzüge des Markenrechts}

Die Marke ist ein Zeichen, das geeignet ist, Waren oder Dienstleistungen
eines Unternehmens von solchen anderer Unternehmen zu unterscheiden.

Marken können insbesondere Wörter (Siemens), Buchstaben (SBB), Zahlen
(501 -> Lewis), bildliche Darstellungen (Swisscom),
dreidimensionale Formen oder Verbindungen solcher Elemente untereinander
oder mit Farben sein.

\subsection{Form}

\begin{itemize}
\tightlist
\item Wortmarke
\item Bildmarke
\item Kombinierte Wort-/Bildmarke
\item Formmarke
\end{itemize}

\subsection{Unterscheidung nach Zweck}

\begin{description}
	\tightlist
	\item[Individualmarke] Marke eines einzelnen Unternehmens für eine
	bestimmte Ware oder Dienstleistung
	\item[Garantiemarke]  Gewährleistung gewisser Produkteigenschaften
	zB: ZEWO
	\item[Kollektivmarke] Kennzeichnung der Produkte einer Vereinigung
	zB: Fleurop, FMH
\end{description}

\subsection{Schutzvoraussetzungen}

\subsubsection{Absolute Ausschlussgründe}

Wenn eine der folgenden Bedingungen eintritt, kann eine Marke
\textbf{nicht registriert} werden:

\begin{enumerate}
	\tightlist
	\item Zeichen des Gemeingutes
	\begin{itemize}
		\tightlist
		\item Sachbezeichnungen: Marken, die ein Produkt bloss beschreiben oder
		eine Sache als diese Sache benennen.\\
		(z.B. Apple für Äpfel)
		\item Freizeichen: Zu Schabezeichnungen ``degeneriert'' oder Konkurrenten
		sind auf die Verwendung dieser Zeichen angewiesen\\
		(z.B. Fön, Linoleum oder Nylon)
	\end{itemize}
	\item Allgemein notwendige Formen: Formen, die das Wesen der Ware ausmachen
	oder technisch notwendig sind (z.B. Fussball)
	\item Irreführende Zeichen: Falsche Erwartungen betreffend Herkunft,
	Qualität oder geschäftlicher Verhältnisse (z.B. Kübler-Rad)
	\item Zeichen gegen die öffentliche Ordnung/guten Sitten
	\item Wappen und andere öffentliche Zeichen
\end{enumerate}

\subsubsection{Relative Ausschlussgründe}

Die Marke wird auf Antrag eines Dritten (Widerspruch) nicht registriert,
wenn identische oder ähnliche Kennzeichen mit gleichen oder
gleichartigen Produkten vorliegen.

--> Verwechslungsgefahr

\mbox{}\\
Wenn eine Person eine Marke beantragt, ist er selber dafür
verantwortlich, dass seine Markenrechte nicht verletzt werden. Man muss
innerhalb von 3 Monaten klagen.


\subsubsection{Verwechslungsgefahr}

\begin{itemize}
	\tightlist
	\item Arten der Verwechslung
	\begin{description}
		\tightlist
		\item[Optik] Wortlänge und Art der Buchstaben
		\item[Akustik] Silbenmass, Aussprachekadenz, Vokalfolge\\
		zB: ''Tobler-o-rum'' / ''Torero-Rum''
		\item[Sinngehalt] Vermittlung von ''Ersatz für'' oder ''gleich gut wie''
	\end{description}
	\item Kennzeichenstärke: Starke Marken haben grösseren Schutzumfang
	\item Adressaten: Sicht des Kaufentscheidfällers
	\item Erinnerungsvermögen der Adressaten: bei Massenartikeln ist
	Verwechslungsgefahr grösser als bei Spezialprodukten / Aufmerksamkeit im
	Schmuckgeschäft höher als im Warenhaus
	\item Massgebend ist der Gesamteindruck beim durchschnittlichen Abnehmer
\end{itemize}

\emph{Die Hinterlegung einer gleichen Marke für völlig verschiedene
Produkte ist grundsätzlich möglich. Je ähnlicher sich die Produkte sind,
desto stärker müssen sich die Marken voneinander unterscheiden.}

\subsection{Registrierung}
\label{sec:Markenrecht-Registrierung}

Markenschutz ist \textbf{territorial}: Schutz nur in dem Land, in dem
Marke registriert ist. In der Schweiz regelt diese das Eisg. Institut für
Geistiges Eigentum (ige). Die meisten Länder werden jedoch von der
IGE -> World Intellectual Property Organziation verwaltet.

\begin{description}
	\item[Eintragungsprinzip] Die Marke entsteht erst mit der Eintragung
	ins Register (Art. 5 MSchG). Eine Ausnahme bildet die \textbf{notorisch
	bekannte Marke}\\
	Bsp: «Galeries Lafayette» vs. «Lafayette Chocolatier en
	Suisse». Marke war in der Schweiz eigentlich nicht geschützt, bekam jedoch in
	einem Streit recht, da sie weit verbreitet war.
	\item[Spezialitätenprinzip] Marke muss einer oder mehreren Waren oder
	Dienstleistungen zugewiesen werden und ist in der Folge lediglich für
	diese Verwendung geschützt (www.ige.ch). Eine Ausnahme bildet die
	berühmte Marke (Art. 15 MSchG).
	\item[Berühmte Marken] Die Marke ist zwar nur für eine Kategorie
	eingetragen, ist aber so bekannt, dass sie für sämtliche
	Markenkategorien geschützt ist. Bsp: Coca Cola.
\end{description}

Markenrechte erhält man für 10 Jahren und können jeweils um weitere 10
Jahre verlängert werden. Ein Markenrecht kann so unendlich lange
verlängert werden (sofern die Gebühren bezahlt werden).

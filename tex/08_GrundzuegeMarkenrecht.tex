\hypertarget{grundzuxfcge-des-markenrechts}{%
\section{Grundzüge des
Markenrechts}\label{grundzuxfcge-des-markenrechts}}

Die Marke ist ein Zeichen, das geeignet ist, Waren oder Dienstleistungen
eines Unternehmens von solchen anderer Unternehmen zu unterscheiden.

Marken können insbesondere Wörter (Siemens), Buchstaben (SBB), Zahlen
(501 -\textgreater{} Lewis), bildliche Darstellungen (Swisscom),
dreidimensionale Formen oder Verbindungen solcher Elemente untereinander
oder mit Farben sein.

\hypertarget{form}{%
\subsection{Form}\label{form}}

\begin{itemize}
\tightlist
\item
  Wortmarke
\item
  Bildmarke
\item
  Kombinierte Wort-/Bildmarke
\item
  Formmarke
\end{itemize}

\hypertarget{zweck}{%
\subsection{Zweck}\label{zweck}}

\begin{itemize}
\tightlist
\item
  \textbf{Individualmarke}: Marke eines einzelnen Unternehmenes für eine
  bestimmte Ware oder Dienstleistung
\item
  \textbf{Garantiemarke}: Gewährleistung gewisser Produkteigenschaften
  zB: ZEWO
\item
  \textbf{Kollektivmarke}: Kennzeichnung der Produkte einer Vereinigung
  zB: Fleurop, FMH
\end{itemize}

\hypertarget{schutzvoraussetzungen}{%
\subsection{Schutzvoraussetzungen}\label{schutzvoraussetzungen}}

\hypertarget{absolute-ausschlussgruxfcnde}{%
\subsubsection{Absolute
Ausschlussgründe}\label{absolute-ausschlussgruxfcnde}}

Wenn eine der folgenden Bedingungen eintritt, kann eine Marke
\textbf{nicht registriert} werden.

\begin{enumerate}
\def\labelenumi{\arabic{enumi}.}
\tightlist
\item
  Zeichen des Gemeingutes

  \begin{itemize}
  \tightlist
  \item
    Sachbezeichnungen: Marken, die ein Produkt bloss beschreiben oder
    eine Sache als diese Sache bennen (z.B. Apple für Äpfel)
  \item
    Freizeichen: Zu Schabezeichnungen ``degeneriert'' oder Konkurrenten
    sind auf die Verwendung dieser Zeichen angewiesen (z.B. Fön,
    Linoleum oder Nylon)
  \end{itemize}
\item
  Allgemein notwendige Formen: Formen, die das Wesen der Ware ausmachen
  oder technisch notwendig sind (z.B. Fussball)
\item
  Irreführende Zeichen: Falsche Erwartungen betreffend Herkunft,
  Qualität oder geschäftlicher Verhältnisse (z.B. Kübler-Rad)
\item
  Zeichen gegen die öffentliche Ordnung/guten Sitten
\item
  Wappen und andere öffentliche Zeichen
\end{enumerate}

\hypertarget{relative-ausschlussgruxfcnde}{%
\subsubsection{Relative
Ausschlussgründe}\label{relative-ausschlussgruxfcnde}}

Die Marke wird auf Antrag eines Dritten (Widerspruch) nicht registriert,
wenn identische oder ähnliche Kennzeichen mit gleichen oder
gleichartigen Produkten vorliegen.

--\textgreater{} Verwechslungsgefahr

Wenn eine Person eine Marke beantragt, ist er selber dafür
verantwortlich, dass seine Markenrechte nicht verletzt werden. Man muss
innerhalb von 3 Monaten klagen.

\hypertarget{verwechslungsgefahr}{%
\subsubsection{Verwechslungsgefahr}\label{verwechslungsgefahr}}

Arten der Verwechslung (Checkliste für Prüfung! alternativ) - Optik:
Wortlänge und Art der Buchstaben - Akustik: Silbenmass,
Aussprachekadenz, Vokalfolge zB: «Tobler-o-rum» / «Torero-Rum» -
Sinngehalt: Vermittlung von «Ersatz für» oder «gleich gut wie» zB:
«Kamillosan» / «Kamillan» / «Kamillon» - Kennzeichenstärke: Starke
Marken haben grösseren Schutzumfang - Adressaten: Sicht des
Kaufentscheidfällers - Erinnerungsvermögen der Adressaten: bei
Massenartikeln ist Verwechslungsgefahr grösser als bei Spezialprodukten
/ Aufmerksamkeit im Schmuckgeschäft höher als im Warenhaus - Massgebend
ist der Gesamteindruck beim durchschnittlichen Abnehmer

\emph{Die Hinterlegung einer gleichen Marke für völlig verschiedene
Produkte ist grundsätzlich möglich. Je ähnlicher sich die Produkte sind,
desto stärker müssen sich die Marken voneinander unterscheiden.}

\hypertarget{registrierung}{%
\subsection{Registrierung}\label{registrierung}}

Markenschutz ist \textbf{territorial}: Schutz nur in dem Land, in dem
Marke registriert ist.

Registrierung \textbf{international}: IGE -\textgreater{} World
Intellectual Property Organziation in Genf.

\textbf{Eintragungsprinzip}: Die Marke entsteht erst mit der Eintragung
ins Register (Art. 5 MSchG). Eine Ausnahme bildet die \textbf{notorisch
bekannte Marke} (Bsp: «Galeries Lafayette» vs. «Lafayette Chocolatier en
Suisse». Marke war in der Schweiz eig. nicht geschützt, bekam jedoch in
einem Streit recht, da sie weit verbreitet war.)

\textbf{Spezialitätenprinzip}: Marke muss einer oder mehreren Waren oder
Dienstleistungen zugewiesen werden und ist in der Folge lediglich für
diese Verwendung geschützt (www.ige.ch). Eine Ausnahme bildet die
berühmte Marke (Art. 15 MSchG).

\textbf{Berühmte Marken}: Die Marke ist zwar nur für eine Kategorie
eingetragen, ist aber so bekannt, dass sie für sämtliche
Markenkategorien geschützt ist. Bsp: Coca Cola.

Markenrechte erhält man für 10 Jahren und können jeweils um weitere 10
Jahre verlängert werden. Ein Markenrecht kann so unendlich lange
verlängert werden (sofern Gebühren bezahlt werden).

\hypertarget{uxfcbungen}{%
\subsection{Übungen}\label{uxfcbungen}}

\textbf{Ausgangslage:} George importiert aus Thailand einige gefälschte
Marken-Uhren, um diese seinen Kollegen zu schenken. Als das Gold auf der
einen Uhr abzublättern beginnt, bringt er diese zum Neuvergolden einem
Bijoutier, welcher die betroffene Uhrenfirma informiert. Diese reicht in
der Folge Strafanzeige ein und stellt den Antrag, die Uhr sei gestützt
auf Art. 68 MSchG einzuziehen und zu vernichten.

\textbf{Kann sich George mit Erfolg dagegen wehren?}\\
Innerhalb der Schweiz ist es nicht verboten, diese gefälschten Artikel
zu privaten Zwecken (Gebrauch) zu benutzen. Nur bei Import, Export oder
Transport durch die Schweiz kann der Hersteller intervenieren.

\textbf{Wie wäre der Fall zu beurteilen, wenn George bereits bei der
Einreise am Flug-hafen Kloten von den Zollbehörden gestoppt worden
wäre?}\\
Dies sind sogenannte Kapilar-Importe (Import von kleinen Mengen
gefälschter Ware). Dies ist verboten!

\textbf{Ausgangslage 2:} Ein junges Zürcher Unternehmen bringt ein
alkoholfreies Bier auf den Markt und beantragt beim IGE die Eintragung
des Werbeslogans «Nicht immer, aber immer öfter!». Gegen den Eintrag
erhebt ein deutscher Bierproduzent Widerspruch, der mit dem gleichen
Spruch wirbt und diesen in Deutschland (nicht aber in der Schweiz) als
Marke geschützt hat.

\textbf{Wird der Widerspruch Erfolg haben?}\\
Der Kläger müsste innerhalb von 3 Monaten reagieren. Und falls dann der
Spruch notorisch bekannt wäre in der Schweiz, hätte der Wiederspruch
Erfolg. Dies müsste man prüfen.

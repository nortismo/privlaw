\section{Strafaspekte}

\subsection{Weshalb Strafrecht?}

\begin{itemize}
	\tightlist
	\item Gewaltmonopol als stabilisierende, kulturelle Leistung. Aber nur, wenn
	monopol demokratisch legitimiert ist und Verfahren und Sanktionen
	voraussehbar sind! Ziel des Strafrechts ist u.a. Abschreckung
	(„Generalprävention``) und individuelle Besserung
	(``Spezialprävention``).
	\item ``\textbf{nulla poena sine lege}'' (keine Strafe one Gesetz)
	\item Bestraft wird nur, wem (kumulativ!) \textbf{tatbestandmässiges},
	\textbf{rechtswidriges} und \textbf{schuldhaftes} Handeln nachgewiesen
	wurde.
\end{itemize}

\subsection{Für die Strafbarkeit sind immer zu prüfen}

\begin{enumerate}
	\tightlist
	\item Menschliche Handlung
	\item Tatbestandsmässig (objektiv/subjektiv)
	\begin{itemize}
		\tightlist
		\item Wurde objektiv betrachtet überhaupt etwas illegales gemacht?
		\item vorsätzlich/eventualvorsätzlich/fahrlässig (grob- oder
		leichtfahrlässig)
	\end{itemize}
	\item Rechtswidrig (Notwehr/Notstand)
	\begin{itemize}
		\tightlist
		\item \textbf{Notstand} bedeutet, wenn man in einer Not-Situation ist und
		sich durch eine strafbare Handlung in Sicherheit bringt. z.B. bei einem
		Sturm im Berg in eine Hütte einbrechen.
		\item \textbf{Notwehr} bedeutet, dass die Handlung stattfand als der Täter
		sein Leben/Gesundheit verteidigte
	\end{itemize}
	\item Schuldhaft (Schuldfähigkeit)
	\begin{itemize}
		\tightlist
		\item schuldunfähig/verminderte Schuldfähigkeit (Kinder sind
		beispielsweise beschränkt haftbar)
	\end{itemize}
	\item Mit Strafe/Sanktion bedroht
\end{enumerate}

\subsection{Deliktarten}
\begin{description}
	\item[Antragsdelikt] Strafverfolgungsorgane werden nur auf Antrag tätig.\\
	Frist von 3 Monaten.
	\item[Offizialdelickt] Strafverfolgungsorgane werden von sich aus tätig.
\end{description}

\subsection{Sanktionen}
\subsubsection{Strafen}

\begin{itemize}
	\tightlist
	\item Freiheisstrafe (Vergehen <= 3 Jahre / Verbrechen >= 3 Jahre)
	\item Geldstrafe
	\item Gemeinnützige Arbeit
\end{itemize}

\subsubsection{Massnahmen}

Sind keine Strafen sondern Massnahmen zur Schutz der Gesellschaft.

\begin{itemize}
	\item therapeutische Massnahme (stationär/ambulant)
	\item Verwahrung
	\item Andere (Berufsverbot/Fahrverbot/Einziehung etc.)
\end{itemize}


\subsection{Strafzumessung}
\begin{itemize}
	\tightlist
	\item Freiheitsstrafen: 3 Tage bis 20 Jahre, z.T. lebenslänglich
	\item Geldstrafen: 1 Tagessatz bis 360 Tagessätze
	\item Gemeinnützige Arbeit: bis 720 Stunden
	\begin{itemize}
		\tightlist
		\item Vorallem im Jugendstrafrecht.
	\end{itemize}
	\item Busse: grundsätzlich bis 10'000.--- CHF, wenn gesetzlich vorgesehen
	aber auch höher
	\item Eine bedingte Haftstrafe kann mit einer unbedingten Geldstrafe/Busse
	verbunden werden!
\end{itemize}

Eine \textbf{bedingte} Haftstrafe kann mit einer unbedingten Geldstrafe/Busse
verbunden werden! Mehrere Taten (bspw. Einbruch = Hausfriedensbruch + Diebstahl)
wirken strafschärfend.

\subsubsection{Bedingt vs.~unbedingt}

Eine \textbf{unbedingte} Strafe wird vollzogen. Eine \textbf{bedingte}
Strafe wird nur vollzogen, wenn in einer Frist nochmals eine Tat
begannen wird. Die Handlung muss dabei nicht im selben Feld bestehen.

\subsection{Ablauf Strafverfahren}

\begin{itemize}
	\tightlist
	\item Polizei/Staatsanwaltschaft (StA) wird aktiv
	\item Untersuchungsleitun durch StA
	\item StA hat Aufgabe, belastende \& entlastende Aspekte zu untersuchen (im
	Zweifel klagt er jedoch an!)
	\item StA entscheidet, ob der Fall zur Beurteilung an Strafgericht
	überwiesen wird
	\item StA hat in einfachen Fällen Strafkompetenz
\end{itemize}

\subsection{Typische Cyber-Delikte}
\label{sec:CD-Overview}

\begin{itemize}
	\tightlist
	\item Unbefugte Datenbeschaffung (143 StGB)
	\item Unbefugtes Eindringen („Hacken``) in Datenverarbeitungssystem (143bis
	StGB)
	\item Datenbeschädigung (Art. 144bis StGB)
	\item Betrügerischer Missbrauch einer Datenverarbeitungsanlage (Art. 147
	StGB)
	\item Herstellen und Inverkehrbringen von Materialien zur unbefugten
	Entschlüsselung codierter Angebote (Art. 150Bis StGB)
	\item Verletzung des Fabrikations- oder Geschäftsgeheimnisses (Art. 162
	StGB)
	\item Ehrverletzungen (173ff StGB)
	\item Verletzung des Schriftgeheimnisses (179 StGB)
	\item Unbefugtes Beschaffen von Personendaten (179novies StGB)
	\item Pornografie (197 StGB)
	\item Störung von Betrieben, die der Allgemeinheit dienen (239 StGB)
	\item Rassendiskriminierung (261 bis StGB)
	\item Wirtschaftlicher Nachrichtendienst (273 StGB)
\end{itemize}

\subsubsection{Unbefugte Datenbeschaffung (Art. 143 StGB)}
\label{sec:CD-Datenbeschaffung}
\begin{enumerate}
	\tightlist
	\item Wer in der Absicht, sich oder einen andern unrechtmässig zu
	bereichern, sich oder einem andern elektronisch oder in
	vergleichbarer Weise gespeicherte oder übermittelte Daten
	beschafft, die nicht für ihn bestimmt und gegen seinen unbefugten
	Zugriff besonders gesichert sind, wird mit Freiheitsstrafe bis zu fünf
	Jahren oder Geldstrafe bestraft.
	\item Die unbefugte Datenbeschaffung zum Nachteil eines Angehörigen
	oder Familiengenossen wird nur auf Antrag verfolgt.
\end{enumerate}

\subsubsection{Unbefugtes Eindringen in ein Datenverarbeitungssystem (Art. 143bis StGB)}
\label{sec:CD-Eindringen}
\begin{enumerate}
	\tightlist
	\item Wer auf dem Wege von Datenübertragungseinrichtungen
	unbefugterweise in ein fremdes, gegen seinen Zugriff besonders gesichertes
	Datenverarbeitungssystem eindringt, wird, auf Antrag, mit Freiheitsstrafe
	bis zu drei Jahren oder Geldstrafe bestraft.
	\item Wer Passwörter, Programme oder andere Daten, von denen er weiss oder
	annehmen muss, dass sie zur Begehung einer strafbaren Handlung gemäss
	Absatz 1 verwendet werden sollen, in Verkehr bringt oder zugänglich
	macht, wird mit Freiheitsstrafe bis zu drei Jahren oder Geldstrafe bestraft.
\end{enumerate}

\subsubsection{Datenbeschädigung (Art. 144bis StGB)}
\label{sec:CD-Datenbeschädigung}
\begin{enumerate}
	\tightlist
	\item Wer unbefugt elektronisch oder in vergleichbarer Weise gespeicherte
	oder übermittelte Daten verändert, löscht oder unbrauchbar macht, wird,
	auf Antrag, mit	Freiheitsstrafe bis zu drei Jahren oder Geldstrafe bestraft.
	\\
	\\
	Hat der Täter einen grossen Schaden verursacht, so kann auf Freiheitsstrafe
	von einem Jahr bis zu fünf Jahren erkannt werden.
	Die Tat wird von Amtes wegen verfolgt.
	\item Wer Programme, von denen er weiss oder annehmen muss, dass sie zu den
	in Ziffer 1 genannten Zwecken verwendet werden sollen, herstellt, einführt,
	in Verkehr bringt, anpreist, anbietet oder sonst wie zugänglich macht oder
	zu ihrer Herstellung Anleitung gibt, wird mit Freiheitsstrafe bis zu drei
	Jahren oder Geldstrafe bestraft.\\
	\\
	Handelt der Täter gewerbsmässig, so kann auf Freiheitsstrafe von einem Jahr bis zu
	fünf Jahren erkannt werden.
\end{enumerate}

\subsubsection{Betrügerischer Missbrauch einer Datenverarbeitungsanlage
(Art. 147 StGB)}
\label{sec:CD-Missbrauch}
\begin{enumerate}
	\tightlist
	\item Wer in der Absicht, sich oder einen andern unrechtmässig zu bereichern,
	durch unrichtige, unvollständige oder unbefugte Verwendung von Daten oder
	in vergleichbarer Weise auf einen elektronischen oder vergleichbaren
	Datenverarbeitungs- oder Datenübermittlungsvorgang einwirkt und dadurch
	eine Vermögensverschiebung zum Schaden eines andern herbeiführt oder
	eine Vermögensverschiebung unmittelbar darnach verdeckt, wird mit
	Freiheitsstrafe bis zu fünf Jahren oder Geldstrafe bestraft.
	\item Handelt der Täter gewerbsmässig, so wird er mit Freiheitsstrafe bis
	zu zehn Jahren oder Geldstrafe nicht unter 90 Tagessätzen bestraft.
	\item Der betrügerische Missbrauch einer Datenverarbeitungsanlage zum
	Nachteil eines Angehörigen oder Familiengenossen wird nur auf Antrag verfolgt.
\end{enumerate}

\subsubsection{Herstellen \& Inverkehrbringen von Materialien zur unbefugten
Entschlüsselung codierter Angebote (Art. 150bis StGB)}
\label{sec:CD-Herstellen}
\begin{enumerate}
	\tightlist
	\item Wer Geräte, deren Bestandteile oder Datenverarbeitungsprogramme,
	die zur unbefugten Entschlüsselung codierter
	Rundfunkprogramme oder Fernmeldedienste bestimmt und
	geeignet sind, herstellt, einführt, ausführt, durchführt, in
	Verkehr bringt oder installiert, wird, auf Antrag, mit Busse
	bestraft.
	\item Versuch und Gehilfenschaft sind strafbar.
\end{enumerate}

\subsubsection{Ehrverletzungs Delikte (Art. 173/174/177 StGB)}
\label{sec:CD-Ehrverletzung}
\begin{description}
	\tightlist
	\item[Üble Nachrede (Art. 173 StGB):] Der Straftatbestand hat zum Ziel,
	jemanden zu bestrafen, der gegenüber Dritten über eine andere Person
	\textbf{rufschädigende vorsätzlich wahre} (oder unvorsätzlich, d.h. gutgläubig
	unwahre) \textbf{Äusserungen tätigt oder weiterverbreite}t. Der Täter kann sich
	allerdings entlasten und bleibt straflos, wenn ihm der sogenannte
	\textbf{Entlastungsbeweis} (wahr \& öffentliches Interesse) gelingt.
	\item[Verleumdung (Art. 174 StGB)] Die Verleumdung ist üble Nachrede
	\textbf{wider besseren Wissens}. Der Täter beschuldigt oder verdächtigt eine
	Person gegenüber Dritten eines unehrenhaften Verhaltens oder anderer
	rufschädigender Tatsachen, die in Wirklichkeit nicht bestehen und somit
	unwahr sind. Der Entlastungsbeweis ist nicht möglich.
	\item[Beschimpfung (Art. 177 StGB)] Der Beschimpfung macht sich strafbar,
	wenn jemand in anderer Weise – d.h. nicht durch üble Nachrede oder
	Verleumdung – \textbf{durch Wort, Schrift, Bild, Gebärde oder Tätlichkeiten
	in der Ehre angegriffen} wird.
\end{description}

\subsection{Tipps}

\begin{itemize}
	\tightlist
	\item Bei Antragsdelikten nicht warten – \textbf{Frist von 90 Tagen!}
	\item Strafanzeige vs. zivilrechtliche Klagen gut abwägen
	(Zeit/Kosten/Nebenfolgen)
	\item Nur \textbf{Privatkläger} erhält Informationen aus der Untersuchung!
	\item Was einmal „unglücklich“ formuliert in den Befragungsprotokollen ist,
	lässt sich kaum korrigieren!
	\item i.d.R. ist Kooperation besser, als Aussageverweigerung! Aber als
	Beschuldigte/r muss man sich nicht belasten!
	\item Notfallplan „Hausdurchsuchung“ vorbereiten!
\end{itemize}


\hypertarget{strafaspekte-von-privacy}{%
\section{Strafaspekte von Privacy}\label{strafaspekte-von-privacy}}

Lernziele: - Frist von 90 Tagen für Antragsdelikte kennen! - Ablauf
Strafverfahren kennen - „Praktikertips`` kennen

\hypertarget{weshalb-strafrecht}{%
\subsection{Weshalb Strafrecht?}\label{weshalb-strafrecht}}

\begin{itemize}
\tightlist
\item
  Gewaltmonopol als stabilisierende, kulturelle Leistung. Aber nur, wenn
  monopol demokratisch legitimiert ist und Verfahren und Sanktionen
  voraussehbar sind! Ziel des Strafrechts ist u.a. Abschreckung
  („Generalprävention``) und individuelle Besserung
  (``Spezialprävention``).
\item
  ``\textbf{nulla poena sine lege}'' (keine Strafe one Gesetz)
\item
  Bestraft wird nur, wem (kumulativ!) \textbf{tatbestandmässiges},
  \textbf{rechtswidriges} und \textbf{schuldhaftes} Handeln nachgewiesen
  wurde.
\end{itemize}

\hypertarget{fuxfcr-die-strafbarkeit-sind-immer-zu-pruxfcfen}{%
\subsection{Für die Strafbarkeit sind immer zu
prüfen}\label{fuxfcr-die-strafbarkeit-sind-immer-zu-pruxfcfen}}

\begin{enumerate}
\def\labelenumi{\arabic{enumi}.}
\tightlist
\item
  Menschliche Handlung
\item
  Tatbestandsmässig (objektiv/subjektiv)

  \begin{itemize}
  \tightlist
  \item
    Wurde objektiv betrachtet überhaupt etwas illegales gemacht?
  \item
    vorsätzlich/eventualvorsätzlich/fahrlässig (grob- oder
    leichtfahrlässig)
  \end{itemize}
\item
  Rechtswidrig (Notwehr/Notstand)

  \begin{itemize}
  \tightlist
  \item
    Notstand bedeutet, wenn man in einer Not-Situation ist und sich
    durch eine strafbare Handlung in Sicherheit bringt. z.B. bei einem
    Sturm im Berg in eine Hütte einbrechen.
  \end{itemize}
\item
  Schuldhaft (Schuldfähigkeit)

  \begin{itemize}
  \tightlist
  \item
    schuldunfähig/verminderte Schuldfähigkeit (Kinder sind
    beispielsweise beschränkt haftbar)
  \end{itemize}
\item
  Mit Strafe/Sanktion bedroht
\end{enumerate}

\hypertarget{sanktionen}{%
\subsection{Sanktionen}\label{sanktionen}}

\hypertarget{strafen}{%
\subsubsection{Strafen}\label{strafen}}

\begin{itemize}
\tightlist
\item
  Freiheisstrafe (Vergehen \textless{}= 3 Jahre / Verbrechen
  \textgreater{}= 3 Jahre)
\item
  Geldstrafe
\item
  Gemeinnützige Arbeit
\end{itemize}

\hypertarget{massnahmen}{%
\subsubsection{Massnahmen}\label{massnahmen}}

Sind keine Strafen sondern Massnahmen zur Schutz der Gesellschaft. -
therapeutische Massnahme (stationär/ambulant) - Verwahrung - andere
(Berfusverbot/Fahrverbot/Einziehung etc.)

\hypertarget{strafzumessung}{%
\subsection{Strafzumessung}\label{strafzumessung}}

\begin{itemize}
\tightlist
\item
  Freiheitsstrafen: 3 Tage bis 20 Jahre, z.T. lebenslänglich
\item
  Geldstrafen: 1 Tagessatz bis 360 Tagessätze
\item
  Gemeinnützige Arbeit: bis 720 Stunden

  \begin{itemize}
  \tightlist
  \item
    Vorallem im Jugendstrafrecht.
  \end{itemize}
\item
  Busse: grundsätzlich bis 10'000.--- CHF, wenn gesetzlich vorgesehen
  aber auch höher
\item
  Eine bedingte Haftstrafe kann mit einer unbedingten Geldstrafe/Busse
  verbunden werden!
\item
  Mehrere Taten (z.B. „Einbruch`` = Hausfriedensbruch und Diebstahl)
  wirken strafschärfend (erhöhter Strafrahmen, aber nicht zusammenzählen
  der Einzelstrafen!)
\item
  Begrifflichkeit: Strafminderung/Straferhöhung ist nicht Strafschärfung
\end{itemize}

\hypertarget{bedingt-vs.unbedingt}{%
\subsubsection{Bedingt vs.~unbedingt}\label{bedingt-vs.unbedingt}}

Eine \textbf{unbedingte} Strafe wird vollzogen. Eine \textbf{bedingte}
Strafe wird nur vollzogen, wenn in einer Frist nochmals eine Tat
begannen wird. Die Handlung muss dabei nicht im selben Feld bestehen.

\hypertarget{ablauf-strafverfahren}{%
\subsection{Ablauf Strafverfahren}\label{ablauf-strafverfahren}}

\begin{itemize}
\tightlist
\item
  Polizei/Staatsanwaltschaft (StA) wird aktiv
\item
  Untersuchungsleitun durch StA
\item
  StA hat Aufgabe, belastende \& entlastende Aspekte zu untersuchen (im
  Zweifel klagt er jedoch an!)
\item
  StA entscheidet, ob der Fall zur Beurteilung an Strafgericht
  überwiesen wird
\item
  StA hat in einfachen Fällen Strafkompetenz
\end{itemize}

\hypertarget{typische-cyber-delikte}{%
\subsection{Typische Cyber-Delikte}\label{typische-cyber-delikte}}

\emph{Details dazu in den Folien}

\begin{itemize}
\tightlist
\item
  Unbefugte Datenbeschaffung (143 StGB)
\item
  Unbefugtes Eindringen („Hacken``) in Datenverarbeitungssystem (143bis
  StGB)
\item
  Datenbeschädigung (Art. 144bis StGB)
\item
  Betrügerischer Missbrauch einer Datenverarbeitungsanlage (Art. 147
  StGB)
\item
  Herstellen und Inverkehrbringen von Materialien zur unbefugten
  Entschlüsselung codierter Angebote (Art. 150Bis StGB)
\item
  Verletzung des Fabrikations- oder Geschäftsgeheimnissess (Art. 162
  StGB)
\item
  Ehrverletzungen (173 ff StGB)
\item
  Verletzung des Schriftgeheimnisses (179 StGB)
\item
  Unbefugtes Beschaffen von Personendaten (179novies StGB)
\item
  Pornografie (197 StGB)
\item
  Störung von Betrieben, die der Allgemeinheit dienen (239 StGB)
\item
  Rassendiskriminierung (261bis StGB)
\item
  Wirtschaftlicher Nachrichtendienst (273 StGB)
\end{itemize}

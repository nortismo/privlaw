\section{Designrecht}

\subsection{Rechtsgrundlage}
\label{sec:Designrecht-Grundlage}

\paragraph{Art. 1 DesG}
Gestaltungen von Erzeugnissen oder Teilen von Erzeugnissen, die namentlich durch die Anordnung von Linien, Flächen, Konturen oder Farben oder durch das verwendete Material charakterisiert sind.\\

Designs stellen die bestimmte äussere Formgebung von etwas -
Zweidimensionalem (Muster) oder von etwas\\
zB: Gestaltung eines Stoffmusters, eines Uhrenzifferblatts oder einer
Flaschenetikette - Dreidimensionalem (Modell) dar\\
zB: die Form einer Zahnbürste, einer Lampe oder eines Stuhls


\subsection{Schutzvoraussetzungen}

\begin{enumerate}
	\tightlist
	\item Neuheit
	\begin{itemize}
		\tightlist
		\item In der Schweiz bisher nicht bekannt
		\item Keinem offenen, zahlenmässig unbeschränkten Personenkreis
		präsentiert (innerhalb 12 Monate nach Markteintritt muss das Deisgn
		registiert werden)
	\end{itemize}
	\item Eigenart
	\begin{itemize}
		\tightlist
		\item Unterscheidung von bisherigen ähnlichen Gestaltungen
		\item Geringere Anforderung an Originalität, als beim URG
	\end{itemize}
	\item Nicht gesetzeswidrig oder anstössiger Natur
\end{enumerate}

\subsubsection{Ausschlussgründe}

Die Merkmale des Designs dürfen nicht ausschliesslich durch die
technische Funktion des Erzeugnisses bedingt sein.\\
--> Formgebung darf nicht durch die technische Funktion
gegeben sein.

\subsection{Registrierung}
\label{sec:Designrecht-Registrierung}

\begin{description}
	\tightlist
	\item[Schweiz] Eidgenössisches Institut für Geistiges Eigentum 
	(ige). Kosten: 200 CHF
	\item[International] World Intellectual Property Organization
	(wipo)
\end{description}

Das Designrecht entsteht mit der Eintragung im Design-Register
(Art. 5 DesG).


\subsubsection{Hinterlegungspriorität}
\label{sec:Designrecht-Hinterlegungsprioritaet}

Das Designrecht steht demjenigen zu, der das Design zuerst hinterlegt
(Art. 6 DesG).

Zur Hinterlegung berechtigt ist diejenige Person, die das Design
entworfen hat, deren Rechtsnachfolgerin oder eine Drittperson, welcher
das Recht aus einem andern Rechtsgrund gehört.\\
Gemeinschaftliche Hinterlegung bei gemeinsamen Design möglich.

\subsection{Schutzdauer}

5 Jahre vom Datum der Hinterlegung an. Möglichkeit der Verlängerung um
4 Schutzperioden von je 5 Jahren. Maximaler Designschutz: 25 Jahre

\subsection{Rechtsschutz}

\begin{itemize}
	\tightlist
	\item Beschwerde an das Bundesverwaltungsgericht
	\begin{itemize}
		\tightlist
		\item gegen Verfügungen des IGE
	\end{itemize}
	\item Zivilverfahren (vgl. Markenrecht)
	\item Strafverfahren
	\item Hilfeleistung der Zollverwaltung
\end{itemize}

\section{Patentrecht}
\label{sec:Patentrecht-Zweck}

Das Patentrecht schützt Erfindungen und gewährt dem Inhaber das
ausschliessliche Recht, die durch das Patent geschützte Erfindung
gewerbsmässig zu nutzen (Art. 1 Abs. 1 PatG).\\
Das bedeutet auch, dass der Patentinhaber allen anderen die Nutzung
derselben Erfindung verbieten darf (Art. 8 PatG).

Der Schutz von Erfindungen soll in zweierlei Hinsicht den technischen
Fortschritt fördern:
\begin{itemize}
	\tightlist
	\item Zum einen bietet das Patent \textbf{Grundlage und Anreiz
	für Investitionen} in Forschung und Entwicklung, da Erfindungen mit dem
	Patent \textbf{geschützt} und \textbf{belohnt} werden.
	\item Zum andern ist die Beschreibung
	der Erfindung in der \textbf{Patentschrift öffentlich zugänglich}, was
	wiederum \textbf{innovationsfördernd} wirken soll.
\end{itemize}


\subsection{Patentarten}

\begin{description}
	\tightlist
	\item[Erzeugnispatent]  Bestimmter Gegenstand mit bestimmten Eigenschaften\\
	(z.B. eine chemische Substanz oder eine Maschine)
	\item[Verfahrenspatent] Zeitliches Aufeinanderfolgen von Geschehnissen\\
	(z.B. Verfahren zur Herstellung eines chemischen Stoffes oder zur
	Bearbeitung eines Metalls)
\end{description}


\subsection{Rechtsgrundlage}
\label{sec:Patentrecht-Rechtsgrundlage}

\begin{itemize}
\tightlist
\item Für neue gewerblich anwendbare Erfindungen werden Erfindungspatente
erteilt.
\item Was sich in nahe liegender Weise aus dem Stand der Technik (Art. 7
Abs. 2) ergibt, ist keine patentierbare Erfindung.
\item Die Patente werden ohne Gewährleistung des Staates erteilt.
\end{itemize}


\subsection{Schutzvoraussetzungen}

\begin{enumerate}
	\tightlist
	\item  Gewerbliche Anwendbarkeit: Alle Erfindungen, die zur Ausübung einer
	Erwerbstätigkeit angewendet werden können
	\item Neuheit: Nicht Stand der Technik (weltweit!)
	\begin{itemize}
		\tightlist
		\item Ausnahme: Unschädliche Offenbarung (Art. 7b PatG) mit Frist 6 Monate
		\item Offensichtlicher Missbrauch, bspw. Verstoss gegen Geheimhaltung
		\item Zurschaustellung an einer offiziell anerkannten Ausstellung
	\end{itemize}
	\item Erfindung: Lehre, wie sich eine Aufgabe durch den Einsatz von
	Naturstoffen und/oder Naturkräften wiederholbar lösen lässt.
	\begin{itemize}
		\tightlist
		\item ''Lösung für ein technisches Problem oder technische Lösung für ein
		Problem''
		\item Für Fachmann nicht naheliegend\\
		Als naheliegend gilt eine Erfindung, wenn eine durchschnittliche
		Fachperson, die mit derselben Problemstellung konfrontiert würde,
		ausgehend vom nächstliegenden Stand der Technik, dieselbe Lösung
		mittels der im betreffenden Gebiet bekannten oder logischen
		Vorgehen, d.h. ohne eigene erfinderische Leistung, aus dem Stand der
		Technik am Anmelde- oder Prioritätsdatum herleiten könnte bzw.
		würde.
	\end{itemize}
\end{enumerate}


\subsection{Was ist nicht patentierbar?}

\begin{itemize}
	\tightlist
	\item Nicht patentierbar sind Ideen, Konzepte, Entdeckungen,
	wissenschaftliche Theorien und mathematische Methoden.
	\item Computerprogramme «als solche » sind ebenfalls nicht patentierbar,
	hingegen können computerimplementierte Erfindungen unter Umständen
	patentierbar sein wie z.B. wie das Betriebssystem Windows.
	\item Anmeldung eines Computerprogramms beim Europäischen Patentamt (Art. 52
	Abs. 2 EPÜ) in Ausnahmefällen möglich, z.B. wenn das Computerprogramm
	«(mehrfachen) technischen Charakter» aufweist.
	\item Ebenfalls von der Patentierung ausgeschlossen sind Erfindungen, deren
	Verwertung gegen die öffentliche Ordnung oder die guten Sitten
	verstösst, wie dies z.B. beim Klonen von Menschen der Fall ist.
\end{itemize}


\subsection{Gesetzlice Begrenzung}
\label{sec:Patentrecht-Begrenzug}

Das Patentgesetz listet eine Reihe von Nutzungsarten auf, welche auch
mit Bezug auf patentierte Erfindungen, explizit zulässig sind (Art. 9
PatG):
\begin{itemize}
	\tightlist
	\item Handlungen im privaten Bereich zu nicht gewerblichen Zwecken
	\item Handlungen zu Forschungs- und Versuchszwecken, die der Gewinnung von
	Erkenntnissen über den Gegenstand der Erfindung, einschliesslich seiner
	Verwendungen, dienen (Forschungsprivileg)
	\item Benützung der Erfindung zu
	Unterrichtszwecken an Lehrstätten
\end{itemize}

\subsection{Arbeitnehmererfindungen}
\label{sec:Patentrecht-Arbeitnehmererfindungen}
\begin{description}
	\item[Diensterfindungen] Sind Erfindungen des Arbeitnehmers, die er
	bei Ausübung einer dienstlichen Tätigkeit und in Erfüllung einer
	vertraglichen Pflichten macht oder an deren Hervorbringung er mitwirkt
	(Art. 332 Abs. 1 OR).
	\begin{itemize}
		\item Eigentum bei Arbeitgeber.
		\item Es besteht kein Anspruch auf finanzielle Entschädigung
		durch den Arbeitgeber.
	\end{itemize}
	\item[Gelegenheitserfindungen] Sind Erfindungen, die der Arbeitnehmer
	in Ausübung der dienstlichen Tätigkeit macht, dies jedoch ausserhalb
	seiner vertraglichen Pflichten.
	\begin{itemize}
		\tightlist
		\item Ohne Vereinbarung: Rechte beim Erfinder (Arbeitnehmer) (Art. 332 Abs.2
		OR).
		\item Mit vorheriger (!) Vereinbarung: Erwerb durch Arbeitgeber für
		angemessene Entschädigung möglich (Art. 332 Abs. 2 OR). Frist 6
		Monate.
		\item Umstände zur Bestimmung der angemessenen Entschädigung
		\item der wirtschaftliche Wert der Arbeitnehmererfindung
		\item die Mitwirkung des Arbeitgebers
		\item die Inanspruchnahme von betrieblichen Hilfspersonen und die
		Aufwendungen des Arbeitnehmers
		\item seine Stellung im Betrieb (Art. 332 Abs. 4 OR).
	\end{itemize}
\end{description}

\subsection{Alternativen}
\begin{itemize}
	\tightlist
	\item \textbf{Geheimhaltung}, z. B. wenn die Erfindung am fertigen Produkt
	nicht nachvollziehbar ist.
	\item In Branchen mit sehr schnellen Entwicklungszyklen ist ein
	\textbf{Patentschutz} möglicherweise \textbf{nicht notwendig}.
	Denn bis die Konkurrenz das eigene Produkt kopiert, bringen Sie bereits die
	nächste Produktgeneration auf den Markt.
	\item \textbf{Wollen Sie Ihre Erfindung nicht schützen},
	aber gleichzeitig verhindern, dass Dritte diese Erfindung patentieren
	lassen, können Sie sie \textbf{veröffentlichen}.
	Damit ist sie nicht mehr neu und somit nicht mehr patentierbar.
	Falls Sie Ihre Erfindung im \textbf{Internet} publizieren wollen,
	müssen Sie sicher stellen, dass Sie das Publikationsdatum zu einem späteren
	Zeitpunkt belegen können.
\end{itemize}

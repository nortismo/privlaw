\section{Einführung ins Recht und dessen Privacy-Aspekte}

\subsection{Recht im technischen Umfeld}

Das Recht legt auch oft einen \textbf{Rahmen} des rechtlich Verlangten
fest (was ist maximal zulässig, was wird minimal verlangt?). Also
\textbf{Aufklärung \& Festlegung von Standards und Verpflichtungen und
Verantwortlichkeiten}. Die Standards werden zwar nicht vom Gesetz
festgelegt, aber es wird darauf verwiesen.

\emph{ABER: Standards/Best Practices klären oft nicht ausreichend alle
rechtlichen Fragen}

\subsection{Recht als ``Risk Management''}

Risiken müssen nicht nur technisch und organisatorisch behandelt werden,
sondern es müssen auch \textbf{rechtliche Massnahmen} getroffen werden.
\textbf{Juristische Probleme können technisch behandelt werden} und
umgekehrt können technische Probleme juristisch gelöst werden.

\emph{Rechtliche Unterstützung möglichst früh im Projekt einschalten
Das Management ist dabei verantwortlich für die Einhaltung von Vorschriften
zu organisieren und zu kontrollieren.}

\subsection{Weshalb Privacy \& Datenschutz?}

\textbf{Personenbezogene Daten in den falschen Händen können eine Person
in vielfältiger Form gefährden \& verletzen}. Vertraulichkeit und das
\textbf{``Recht auf Vergessen''} ist elementar. Inhaber von grossen
Mengen an personenbezogenen Daten haben eine machtvolle Stellung ohne
Kontrolle. Einer der Hauptaufgaben eines Staates ist es, seine Bürger zu
beschützen. Doch wie kann er das bzgl. Missbrauch von Personendaten?\\
Zusätzlich haben personenbezogene Daten einen gewissen finanziellen Wert.
Wer ist berechtigt, diese Daten zu nutzen und zu verwerten?
Unsere Daten werden einmal verkauft, aber ökonomisch fortgesetzt genutzt.

\mbox{}\\
\emph{Die Weitergabe von persönlichen Daten in einem Unternehmen ist
auch eine Form von Mobbing und/oder schlechtem Management.}

\subsection{Zentrale (Obligationenrechtliche) Frage}

\begin{enumerate}
\def\labelenumi{\arabic{enumi})}
\tightlist
\item \textbf{WER} will
\item von \textbf{WEM}
\item \textbf{WAS}
\item \textbf{WORAUS}?
\end{enumerate}

\emph{Obligationenrechtlich = Fragen nach Rechtsansprüchen, z.B. aus Vertrag,
unerlaubter Handlung oder ungerechtfertigter Bereicherung.}

\subsection{Site, Moral und Recht}

Rechtliche Verbindlichkeit ist für alle einzufordern (\textbf{grösster
gemeinsamer Nenner} einer vielfältigen Gemeinschaft)


\subsection{Die Juristische Argumentation (!)}

Eine juristische Argumentation sieht immer so aus, dass man eine
Behauptung aufstellt und diese anschliessend begründet. Die Begründung
beruht auf einem Beweis oder einem Gesetzesartikel.

\mbox{}\\
\textbf{Behauptung} wird durch Grundlage (\textbf{Gesetzesartikel}) und
notwendigen \textbf{Beweis} gestützt.\\
\\
\textit{oder}\\ 
\\
Gestützt auf Grundlage (\textbf{Gesetzesartikel}) und notwendigem
\textbf{Beweis} ergibt sich die \textbf{Schlussfolgerung}.


\subsection{Rechtsordnung unter verschiedenen Blickwinkeln}
%
\begin{itemize}
\tightlist
\item Rang (Verfassung, Gesetz, Verordnung)
\item erlassendem Gemeinwesen (Bundesrecht, kantonales- und
Gemeinderecht)
\item Rechtsquelle (geschriebenes Recht, Gewohnheitsrecht,
Gerichtspraxis (Gesetzesauslegung), ZGB)
\item Beteiligten Personen (Privatrecht, öffentliches Recht)
\end{itemize}


\subsection{Hierarchie des Rechts}

\begin{verbatim}
----------------
| Verfassung   |
----------------
       |
----------------
| Gesetze      |
----------------
       |
----------------
| Verordnungen |
----------------
\end{verbatim}

Verordnungen \textbf{sind keine} Verfügungen.\\
Beispiele für Verfügungen: Eine Hochschule besitzt die Verfügung,
Personen als Ingenieur auszuzeichnen.

\subsubsection{Bund/Kantone/Gemeinden}

Kantone stehen in der Gesetzgebungsmacht über dem Bund! Kantone aber
über den Gemeinden

\subsection{Privat- vs. öffentliches Recht}

\begin{description}
	\item[Privatrecht] Zwischen natürlichen bzw. juristischen Personen.\\
	Geprägt vom Grundsatz der \textbf{Koalitions- und Vertragsfreiheit}.
	\item[Öffentliches Recht] Zwischen Staat und natürlichen/juristischen
	Personen.\\
	\textbf{Legalitätsprinzip (Gewaltenkontrolle)}\index{Legalitätsprinzip}
	\index{Gewaltenkontrolle}:
	Der Staat darf nur handeln,
	wenn eine gesetzliche Grundlage besteht!
\end{description}

\subsection{Anwendbarkeit von ausländischem Recht}

\begin{itemize}
	\tightlist
	\item Nebst Völkerrecht, (bi-/multilateralen) internationalen Verträgen ist
	das IPRG (Gesetz über das internationale Privatrecht)\index{IPRG}
	``Hauptschnittstelle'' zwischen CH- und ausländischem Recht.
	\item IPRG regelt, wann welches Recht (CH oder Ausland) anwendbar ist und
	welche Richter zuständig sein sollen.
\end{itemize}

\subsection{Instanzenzug}

Zivil-, Verwaltungs- und Strafgerichte haben unterschiedliche
Verfahren. In allen Rechtsbereichen gibt es jedoch drei Instanzen:

\begin{enumerate}
	\tightlist
	\item Bezirksgericht
	\item Kantonsgericht
	\item Bundesgericht
\end{enumerate}

\subsection{Verschiedene Prozessverfahren und deren Eigenheiten}

\subsubsection{Zivilprozess}

\begin{itemize}
\tightlist
\item Prozesse müssen vor dem örtlich/sachlich \textbf{zuständigen} Gericht
  ausgetragen werden.
\item Ein \textbf{Gerichtskostenvorschuss} ist nötig, um ein Zivilverfahren
  einzuleiten. Der Vorschuss ist abhängig vom Streitwert.
\item Das Gericht sucht keine Beweise. Der \textbf{behauptete Anspruch muss
  bewiesen werden können}.
\item Wer verliert, muss die \textbf{Gerichtskosten sowie Parteikosten der
  andern Seite übernehmen}.
\item Wer einen Forderungsprozess gewinnt, hat das Geld \textbf{noch
  nicht}\ldots{}
\end{itemize}


\subsubsection{Strafverfahren}

\begin{itemize}
	\tightlist
	\item Geregelt im StGB \& StPO. Polizei unterliegt überwiegend
	kantonaler Hoheit.
	\item Für Antragsdelikte (bspw. Diebstahl, Ehrverletzung) gilt eine Frist 
	von \textbf{3 Monaten}!
	\item Staatsanwaltschaft leitet Untersuchung und muss \textbf{belastende \&
	entlastende} Aspekte sammeln.
	\item  Die Einsicht als Beteiligter ist Begrenzt. Durch das Einreichen einer
	\textbf{Privatstrafklageverfahren}\index{Privatstrafklageverfahren}
	bekommt man als Opfer erweitertes Einsichtsrecht.
	\item Der Staatsanwaltschaft stellt das Verfahren ein, straft mit max.
	6 Monaten Freiheitsstrafen und/oder 180 Tagessätze oder überweist den
	Fall zur Beurteilung an das Strafgericht.
	\item Als Verurteilter kann das Urteil auch an das Strafgericht
	weitergezogen werden.
\end{itemize}


\subsection{Verwaltungsverfahren}

\begin{itemize}
	\tightlist
	\item \textbf{Verfügungen} müssen immer durch die \textbf{richtige Behörde}
	im \textbf{richtigen Verfahren} und unter \textbf{Angabe der
	Rechtsmittels} dagegen erlassen werden. Fehlt eine dieser
	Voraussetzungen, ist die Verfügung nichtig!
	\item Wenn neue Tatsachen auftauchen, ist es möglich eine
	Wiedererwägung/Einsprache gegen die Verfügung zu erheben. -Gegen
	Verfügungen kann i.d.R. Beschwerde innert 10/20/30 Tagen geführt
	werden.
	\item Je nach Gesetzesgrundlage ist ein kantonales Obergericht oder das
	Bundesgericht höchste Instanz.
\end{itemize}

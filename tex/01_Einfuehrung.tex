\hypertarget{einfuxfchrung-ins-recht-und-dessen-privacy-aspekte}{%
\section{Einführung ins Recht und dessen
Privacy-Aspekte}\label{einfuxfchrung-ins-recht-und-dessen-privacy-aspekte}}

\hypertarget{lernziele}{%
\subsection{Lernziele}\label{lernziele}}

Nach dieser Vorlesung - Können Sie die Gewaltenteilung erklären - Können
Sie die Einbettung des Rechts zwischen Technik und Management erklären -
Können Sie zuordnen, ob ein Rechtsverhältnis dem Privatrecht oder dem
öffentlichen Recht untersteht - Wissen Sie, was das Recht auf
informationelle Selbstbestimmung beinhaltet - Kennen Sie die Essentials
von Zivil-, Verwaltungs- und Strafverfahren

\hypertarget{recht-im-technischen-umfeld}{%
\subsection{Recht im technischen
Umfeld}\label{recht-im-technischen-umfeld}}

Das Recht legt auch oft einen \textbf{Rahmen} des rechtlich Verlangten
fest (was ist max. zulässig, was muss minimal gemacht werden?). Also
\textbf{Aufklärung \& Festlegung von Standards und Verpflichtungen und
Verantwortlichkeiten}. Die Standards werden zwar nicht vom Gesetz
festgelegt, aber es wird darauf verwiesen.

\emph{ABER: Standards/Best Practices klären oft nicht ausreichend alle
rechtlichen Fragen}

\hypertarget{recht-als-risk-management}{%
\subsection{Recht als ``Risk
Management''}\label{recht-als-risk-management}}

Risiken müssen nicht nur technisch und organisatorisch behandelt werden,
sondern es müssen auch \textbf{rechtliche Massnahmen} getroffen werden.
\textbf{Juristische Probleme können technisch behandelt werden} und
umgekehrt können technische Probleme juristisch gelöst werden.

\begin{verbatim}
Rechtliche Unterstützung möglichst früh im Projekt einschalten. Das Management ist dabei verantwortlich für die Einhaltung von Vorschriften zu organisieren und zu kontrollieren. 
\end{verbatim}

\hypertarget{weshalb-privacy-datenschutz}{%
\subsection{Weshalb Privacy \&
Datenschutz?}\label{weshalb-privacy-datenschutz}}

\textbf{Personenbezogene Daten in den falschen Händen können eine Person
in vielfältiger Form gefährden} \& verletzen. Vertraulichtkeit und das
\textbf{``Recht auf Vergessen''} ist elementar. Inhaber von grossen
Mengen an personenbezogenen Daten haben eine machtvolle Stellung ohne
Kontrolle. Einer der Hauptaufgaben eines Staates ist es, seine Bürger zu
beschützen. Doch wie kann er das bzgl. Missbrauch von Personendaten?

\emph{Die Weitergabe von persönlichen Daten in einem Unternehmen ist
auch eine Form von Mobbing und / oder schlechtem Management.}

\hypertarget{zentrale-obligationenrechtliche-frage}{%
\subsection{Zentrale (Obligationenrechtliche)
Frage}\label{zentrale-obligationenrechtliche-frage}}

\begin{enumerate}
\def\labelenumi{\arabic{enumi})}
\tightlist
\item
  \textbf{WER} will
\item
  von \textbf{WEM}
\item
  \textbf{WAS}
\item
  \textbf{WORAUS}?
\end{enumerate}

\begin{verbatim}
Obligationenrechtlich = Fragen nach Rechtsansprüchen, z.B. aus Vertrag, unerlaubter Handlung oder ungerechtfertigter Bereicherung
\end{verbatim}

\hypertarget{die-juristische-argumentation}{%
\subsection{Die Juristische Argumentation
(!)}\label{die-juristische-argumentation}}

Eine juristische Argumentation sieht immer so aus, dass man eine
Behauptung aufstellt und diese anschliessend begründet. Die Begründung
beruht auf einem Beweis oder einem Gesetzesartikel.

\begin{verbatim}
Behauptung wird durch Grundlage (Gesetzesartikel) und notwendigen Beweis gestützt. 

Gestützt auf Grundlage (Gesetzesartikel) und notwendigem Beweis ergibt sich die Schlussfolgerung.
\end{verbatim}

\hypertarget{rechtsordnung-unter-verschiedenen-blickwinkeln}{%
\subsection{Rechtsordnung unter verschiedenen
Blickwinkeln}\label{rechtsordnung-unter-verschiedenen-blickwinkeln}}

\begin{itemize}
\tightlist
\item
  Rang (Verfassung, Gesetz, Verordnung)
\item
  erlassendem Gemeinwesen (Bundesrecht, kantonales- und Gemeinderecht)
\item
  Rechtsquelle (geschriebenes Recht, Gewohnheitsrecht, Gerichtspraxis,
  ZGB)
\item
  Beteiligten Personen (Privatrecht, öffentliches Recht)
\end{itemize}

\hypertarget{hierarchie-des-rechts}{%
\subsection{Hierarchie des Rechts}\label{hierarchie-des-rechts}}

\begin{verbatim}
----------------
| Verfassung   |
----------------
       |
----------------
| Gesetze      |
----------------
       |
----------------
| Verordnungen |
----------------
\end{verbatim}

Verordnungen \textbf{sind keine} Verfügungen.\\
Beispiele für Verfügungen: Eine Hochschule besitzt die Verfügung,
Personen als Ingenieur auszuzeichnen.

\hypertarget{bundkantonegemeinden}{%
\subsubsection{Bund/Kantone/Gemeinden}\label{bundkantonegemeinden}}

Kantone stehen in der Gesetzgebungsmacht über dem Bund! Kantone aber
über den Gemeinden

\hypertarget{privat--vs.uxf6ffentliches-recht}{%
\subsection{Privat- vs.~öffentliches
Recht}\label{privat--vs.uxf6ffentliches-recht}}

\begin{itemize}
\tightlist
\item
  \textbf{PRIVATRECHT}: Zwischen natürlichen bzw. juristischen
  Personen.\\
  Geprägt vom Grundsatz der \textbf{Koalitions- und Vertragsfreiheit}.
\item
  \textbf{ÖFFENTLICHES RECHT}: Zwischen Staat und
  natürlichen/juristischen Personen.\\
  Legalitätsprinzip (Gewaltenkontrolle): Der Staat darf nur handeln,
  wenn eine gesetzliche Grundlage besteht!
\end{itemize}

\hypertarget{anwendbarkeit-von-ausluxe4ndischem-recht}{%
\subsection{Anwendbarkeit von ausländischem
Recht}\label{anwendbarkeit-von-ausluxe4ndischem-recht}}

\begin{itemize}
\tightlist
\item
  Nebst Völkerrecht, (bi-/multilateralen) internationalen Verträgen ist
  das IPRG (Gesetz über das internationale Privatrecht)
  „Hauptschnittstelle`` zwischen CH- und ausländischem Recht
\item
  IPRG regelt, wann welches Recht (CH oder Ausland) anwendbar ist und
  welche Richter zuständig sein sollen.
\end{itemize}

\hypertarget{instanzenzug}{%
\subsection{Instanzenzug}\label{instanzenzug}}

Zivil-, Verwaltungs- und Strafgerichte haben unter- schiedliche
Verfahren. In allen Rechtsbereichen gibt es jedoch drei Instanzen:\\
\textbf{Bezirksgericht} - \textbf{Kantonsgericht} -
\textbf{Bundesgericht}

\hypertarget{essentials-in-einem-zivilprozess}{%
\subsection{Essentials in einem
Zivilprozess}\label{essentials-in-einem-zivilprozess}}

\begin{itemize}
\tightlist
\item
  Prozesse müssen vor dem örtlich/sachlich \textbf{zuständigen} Gericht
  ausgetragen werden.
\item
  Ein \textbf{Gerichtskostenvorschuss} ist nötig, um ein Zivilverfahren
  einzuleiten. Der Vorschuss ist abhängig vom Streitwert.
\item
  Das Gericht sucht keine Beweise. Der \textbf{behauptete Anspruch muss
  bewiesen werden können}.
\item
  Wer verliert, muss die \textbf{Gerichtskosten sowie Parteikosten der
  andern Seite übernehmen}.
\item
  Wer einen Forderungsprozess gewinn, hat das Geld \textbf{noch
  nicht}\ldots{}
\end{itemize}

\hypertarget{essentials-in-einem-strafverfahren}{%
\subsection{Essentials in einem
Strafverfahren}\label{essentials-in-einem-strafverfahren}}

\begin{itemize}
\tightlist
\item
  Geregelt im StGB \& StPO.
\item
  Für Antragsdelikte (bspw. Diebstahl, Ehrverletzung) gilt eine Frist
  von \textbf{3 Monaten}!
\item
  Staatsanwaldschaft leitet Untersuchung und muss \textbf{belastende \&
  entlastende} Aspekte sammeln.
\item
  Die Einsicht als Beteiligter ist Begrenzt. Durch das Einreichen einer
  \emph{Privatstrafklageverfahren} bekommt man als Opfer erweitertes
  Einsichtsrecht.
\item
  Der Staatsanwaldschaft stellt das Verfahren ein, straft mit max. 6
  Monaten Freiheitsstrafen und/oder 180 Tagessätze oder überweist den
  Fall zur Beurteilung an das Strafgericht.
\item
  Als Verurteilter kann das Urteil auch an das Strafgericht
  weitergezogen werden.
\end{itemize}

\hypertarget{essential-im-verwaltungsverfahren}{%
\subsection{Essential im
Verwaltungsverfahren}\label{essential-im-verwaltungsverfahren}}

\begin{itemize}
\tightlist
\item
  \textbf{Verfügungen} müssen immer durch die \textbf{richtige Behörde}
  im \textbf{richtigen Verfahren} und unter \textbf{Angabe der
  Rechtsmittels} dagegen erlassen werden. Fehlt eine dieser
  Voraussetzungen, ist die Verfügung nichtig!
\item
  Wenn neue Tatsachen auftauchen, ist es möglich eine
  Wiedererwägung/Einsprache gegen die Verfügung zu erheben. -Gegen
  Verfügungen kann i.d.R. Beschwerde innert 10/20/30 Tagen geführt
  werden.
\item
  Je nach Gesetzesgrundlage ist ein kantonales Obergericht oder das
  Bundesgericht höchste Instanz.
\end{itemize}

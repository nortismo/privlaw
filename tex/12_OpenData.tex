\section{Öffentlichkeitsprinzip und Open Data}

Compliance für die öffentliche Verwaltung erfordert:

\begin{itemize}
	\tightlist
	\item Transparenz
	\item Nachvollziehbarkeit
	\item 4-Augen-Prinzip
\end{itemize}

Diese Anforderungen haben zum \textbf{Öffentlichkeitsgesetz (BGÖ)} und der
\textbf{elektronischen Geschäftsverwaltung (GEVER)} geführt.

\subsection{Inhalt des BGÖ}
Alle Personen erhalten danach grundsätzlich Zugang zu jeder Information und
jedem Dokument der Bundesverwaltung. Dies gilt jedoch nicht, wenn
insbesondere die Privatsphäre Dritter verletzt oder die Sicherheit der Schweiz
gefährdet werden kann.\\
In der Schweiz ist dieses Gesetz am 1. Juli 2006 inkraftgetreten.\\

Die meisten Kantone haben selbst auch ein solches Gesetz auf kantonaler Ebene.
Kein solches Gesetz haben aktuell noch die Kantone Thurgau (in Diskussion),
Appenzell Ausserrhoden, Glarus (in Diskussion), Luzern, Obwalden und Nidwalden.\\

Dies ist ein grosser Paradigmenwechsel, denn früher galt:\\
Alle Informationen waren geheim (mit Ausnahmen), kein Anspruch auf Information
über die Verwaltungstätigkeit, kein Anspruch auf Dokumentenzugang,
freies Ermessen der Behörden bei der Gewährung von Ausnahmen.

\subsection{Ziele des BGÖ}

\begin{itemize}
	\tightlist
	\item Kultur der Transparenz in der Verwaltung – Informationsbedürfnisse
	berücksichtigen
	\item Durchsetzbares Recht auf Zugang zu amtlichen Dokumenten für jede
	Person (kein besonderes Interesse muss nachgewiesen werden)
	\item Verbesserung Beziehung Staat und Bürger
	\item Stärkung demokratischer Kontrollrechte
	\item Information als Voraussetzung für politische Mitwirkung
	\item Effizienzgewinn in der Verwaltung
	\item Vorteile für die Wirtschaft
\end{itemize}

\subsection{Geltungsbereich des BGÖ}

Das Gesetz gilt gemäss Art. 5 des BGÖ für \textbf{Amtliche Dokumente, die nach
Inkrafttreten des BGÖ entstanden sind:}
\begin{itemize}
	\tightlist
	\item Information auf beliebigem Informationsträger
	\item Information im Besitz einer Behörde
	\item Zur Erfüllung einer öffentlichen Aufgabe
\end{itemize}

Ausgenommen sind dabei:
\begin{itemize}
	\tightlist
	\item Kommerziell genutzte Dokumente (Geodaten)
	\item Nicht fertiggestellte Dokumente (Entwürfe)
	\item Dokumente für persönlichen Gebrauch (Notizen)
\end{itemize}

Als \textbf{amtliche Dokumente} gelten alle Dokumente welche von der
\textit{Bundesverwaltung}, \textit{Organisationen und Personen des öffentlichen
und privaten Rechts (Post, SBB, SUCA etc.)} und \textit{Parlamentsdienste}
veröffentlicht werden. Es gilt nicht für die \textit{Schweizerische Nationalbank}
sowie die \textit{Eidgenössische Finanzmarktaufsicht} (Art.2 BGÖ).

\subsubsection{Ausnahmen (Art. 7 BGÖ)}
\textbf{Zugang wird eingeschränkt, aufgeschoben oder verweigert:}
\begin{itemize}
	\tightlist
	\item Freie Meinungs- und Willensbildung einer Behörde
	\item Durchführung behördlicher Massnahmen
	\item Innere oder äussere Sicherheit der Schweiz
	\item Aussenpolitische Interessen oder die internationalen Beziehungen
	\item Beziehungen zwischen dem Bund und den Kantonen oder zwischen Kantonen
	\item Wirtschafts-, geld- und währungspolitischen Interessen der Schweiz
	\item Berufs-, Geschäfts- oder Fabrikationsgeheimnisse
	\item Informationen von Dritten freiwillig mitgeteilt.
\end{itemize}

\subsubsection{Interessenabwägung (Art. 6 BGÖ)}
Interessenabwägung zwischen dem Schutz der Privatsphären Dritter und
öffentlichem Interesse am Zugang.

\textbf{Das öffentliche Interesse kann überwiegen:}
\begin{itemize}
	\tightlist
	\item Beim Vorliegen eines besonderen Informationsinteresses der Öffentlichkeit
	(z.B. Korruptionsvorfälle)
	\item Beim Schutz spezifischer öffentlicher Interessen wie öffentliche Ordnung
	und	Sicherheit oder öffentliche Gesundheit
	\item Bei Empfängern von Finanzhilfen oder Abgeltungen
\end{itemize}

\subsection{OpenData}
Offene Daten sind sämtliche Datenbestände, die im Interesse der Allgemeinheit
der Gesellschaft ohne jedwede Einschränkung zur freien Nutzung, zur
Weiterverbreitung und zur freien Weiterverwendung frei zugänglich gemacht
werden.\\

Der Staat hat für die Bürger die Infrastruktur bereit zu stellen. Dazu
gehören zunehmend auch Daten. Dies bietet eingrosses Potential für neue Produkte
\& Dienstleistungen. Grundlage für zahlreiche AI-Anwendungen (als Testdaten,
als Grundlage für Anwendungen).

\subsubsection{OpenData und digitale Transformation}

\begin{itemize}
	\tightlist
	\item opendata.swiss ist das Portal des Bundes \& einiger (weniger) Kantone und
	Gemeinden für öffentlich zugängliche Daten.
	\item Teil der Open-Government-Data-Strategie Schweiz 2014-2018 des
	Bundesrates (e-Government). Bundesrat verabschiedet in den nächsten
	Wochen die Fortsetzung und veröffentlicht Bericht.
	\item Einige Datensätze sind frei verfügbar und dürfen auch kommerziell genutzt
	werden, andere nur mit Einwilligung und Gebühr.
	\item Eher Zurückhaltung der Kantone (Ausnahme z.B. OpenZH) und Gemeinden, da
	man mit den Daten vielleicht Geld verdienen könnte. Kurzfristige Sichtweise.
\end{itemize}
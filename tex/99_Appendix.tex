\section{Gesetzesartikel}

\subsection{Urheberrechtsgesetze (URG)}

\begin{description}
	\tightlist
	\item[Art. 3 URG] Werke zweiter Hand.
	(Siehe \autopageref{sec:Urheberrecht-Schutzvoraussetzugen})

	\item[Art. 6 URG] Als \textbf{Urheber} gilt die \textbf{natürliche Person},
	die das Werk geschaffen hat (\textbf{Schöpferprinzip}).
	(Siehe \autopageref{sec:Urheberrecht-UrheberschaftMiturheberschaft})

	\item[Art. 9.1 URG] Recht auf Erstveröffentlichung
	(Siehe \autopageref{sec:Urheberrecht-Urherberpersönlichkeitsrechte})

	\item[Art. 9.2 URG] Recht auf Anerkennung der Urheberschaft
	(Siehe \autopageref{sec:Urheberrecht-Urherberpersönlichkeitsrechte})

	\item[Art. 11 URG] Recht auf Werkintegrität
	(Siehe \autopageref{sec:Urheberrecht-Urherberpersönlichkeitsrechte})
	
	\item[Art. 16.1 URG] Die Urheberpersönlichkeitsrechte sind unter Lebenden
	nicht übertragbar, von Todes wegen gehen sie jedoch auf die Erben über.
	(Siehe \autopageref{ec:Urheberrecht-Rechtsübergang})

	\item[Art. 17 URG] Automatische Urheberrechtsübertretung bei
	Computerprogrammen die im Rahmen eines Arbeitsvertrages zustande gekommen
	ist.
	(Siehe \autopageref{sec:Urheberrecht-AbhängigeWerkschöpfung})
	
	\item[Art. 19 URG] Grenzen/Schranken des Urheberrechts
	(Siehe \autopageref{sec:Urheberrecht-Schranken})
\end{description}
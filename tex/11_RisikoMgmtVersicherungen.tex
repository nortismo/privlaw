\section{Risikomanagement und Versicherung}

Risiko ist definiert als Eintrittswahrscheinlichkeit(\%) multipliziert mit
dem möglichen Schadensausmass(CHF).\\

Risikomanagement ist Teil der Geschäftsführerverantwortung. Im OR ist sie im
Art. 717 und Art. 754 definiert.

\paragraph{Art. 717 OR IV. Sorgfalts- und Treuepflicht}
\begin{enumerate}
	\tightlist
	\item Die Mitglieder des Verwaltungsrates sowie Dritte, die mit der
	Geschäftsführung befasst sind, müssen ihre Aufgaben mit aller Sorgfalt
	erfüllen und die Interessen der Gesellschaft in guten Treuen wahren.
	\item Sie haben die Aktionäre unter gleichen Voraussetzungen gleich zu
	behandeln.
\end{enumerate}

\paragraph{Art. 754 OR A. Haftung / III. Haftung für Verwaltung,
Geschäftsführung und Liquidation}
\begin{enumerate}
	\tightlist
	\item Die Mitglieder des Verwaltungsrates und alle mit der Geschäftsführung
	oder mit der Liquidation befassten Personen sind sowohl der Gesellschaft als
	den einzelnen Aktionären und Gesellschaftsgläubigern für den Schaden
	verantwortlich, den sie durch absichtliche oder fahrlässige Verletzung ihrer
	Pflichten verursachen.
	\item Wer die Erfüllung einer Aufgabe befugterweise einem anderen Organ
	überträgt, haftet für den von diesem verursachten Schaden, sofern er nicht
	nachweist, dass er bei der Auswahl, Unterrichtung und Überwachung die nach
	den Umständen gebotene Sorgfalt angewendet hat.
\end{enumerate}

\subsection{Versicherungen}

Versicherungen sind eine Möglichkeit das Risiko zu minimieren. Die
Eintrittswahrscheinlichkeit bleibt bestehen, aber das Schadenausmass kann
dadurch reduziert werden.\\

Es gibt folgende Kategorien von Versicherungsdeckung:
\begin{itemize}
	\tightlist
	\item Verschuldens- \& Kausalhaftung (Wer hat den Schaden zu tragen?)
	\item Personenschaden, Sachschaden, Vermögensschaden, Direkter Schaden,
	Indirekter/Folgeschaden (Schadensart)
	\item Versicherte Risiken (Ursachen): Fehlfunktionen, Manipulationen,
	natürliche Risiken
	\item Versicherte Risiken (Objekte): Dritte, Arbeitnehmer, Hardware,
	Datenverlust, Entschädigungen für korrekte Vertragserfüllung,
	IT-Ausfall etc.
\end{itemize}

\subsubsection{Haftpflichtversicherungen}
Grundsatz: Versicherungen zur Deckung von Schäden aus
nicht richtiger Vertragserfüllung enthalten meist eine
Karenzfrist. Wird auf eine Karenzfrist verzichtet, so sind meist
Kürzungen oder Leistungsverweigerung bei schuldhafter
Verletzung der vertraglichen Pflichten die Folge.

\begin{description}
	\item[Karenzfrist] Die Karenzfrist ist die Wartefrist oder die Dauer
	zwischen dem Versicherungsabschluss und dem Moment ab dem Leistungen aus der
	abgeschlossenen Versicherung bezogen werden können.
\end{description}

\subsubsection{Betriebsausfallversicherung}
Eine solche Versicherung deckt Risiken bei einem Betriebsausfall bspw. bei
IT-Problemen. Inklusive:

\begin{itemize}
	\tightlist
	\item Schäden an eigenen Daten (Wiederherstellung)
	\item Schäden durch Hacker, Viren oder leichtfahrlässige Handlungen
	bzw. Unterlassungen von Mitarbeitenden
	\item Haftpflichtansprüche Dritter
\end{itemize}

Schäden aufgrund grober Fahrlässigkeit, Vorsatz oder mangelhafter Weisungen oder
Kontrollen werden von den Versicherungen i.d.R. nicht gedeckt.

\subsection{Haftungsbegrenzungen}
Eine Haftung kann folgendermassen begrenzt werden:
\begin{itemize}
	\tightlist
	\item Vertraglicher Ausschluss
	\item Fristablauf
	\item Höhe des Schadens
	\item Direkten/indirekten Schaden
\end{itemize}

\subsubsection{Möglicher Vertragstext}
„Der Verkäufer ist haftbar für direkten Schaden, den
er grobfahrlässig verursacht, maximal bis zur Höhe
des Betrages der gesamten Projektkosten. Jegliche
übrige Haftung wird im Rahmen des rechtlich
zulässigen ausgeschlossen.“

\subsubsection{Gesetzliche Regelung (Art. 100 OR)}
\begin{enumerate}
	\item Eine zum voraus getroffene Verabredung, wonach die Haftung für
	rechtswidrige Absicht oder grobe Fahrlässigkeit ausgeschlossen sein würde,
	ist nichtig.
	\item Auch ein zum voraus erklärter Verzicht auf Haftung für leichtes Verschulden
	kann nach Ermessen des Richters als nichtig betrachtet werden, wenn der
	Verzichtende zur Zeit seiner Erklärung im Dienst des anderen Teiles stand, oder
	wenn die Verantwortlichkeit aus dem Betriebe eines obrigkeitlich
	konzessionierten Gewerbes folgt.
	\item Vorbehalten bleiben die besonderen Vorschriften über den
	Versicherungsvertrag.
\end{enumerate}
